\documentclass{article}
\author{Shane Pereira}
\usepackage{pdfpages}
\usepackage{apacite}
\title{Medicine Dispenser}

\usepackage{fancyhdr}
\pagestyle{fancy}
\lhead{Medicine Dispenser}
\rhead{Shane Pereira - 0907749}
\rfoot{\LaTeX}


\usepackage[T1]{fontenc}
\usepackage[ttdefault=true]{AnonymousPro}
\renewcommand*\familydefault{\ttdefault}
	
\begin{document}
	\thispagestyle{empty}
	\pagenumbering{gobble}
	\maketitle
	\begin{figure}
		\centering
		\includegraphics[width = 30mm]{Hogoschool-Rotterdam-Logo.pdf}
		\caption{Hogeschool Rotterdam Logo}
		\label{fig:HRO Logo}
	\end{figure}
	
	\newpage
	\tableofcontents
	\newpage
	\pagenumbering{arabic}
	

\section{Exordium}
How much would you value your life if you barely communicated with anyone but your co-workers? This is what is going on in nursing homes across the Netherlands. Lack of staff leads to not having the time to socialise with the clients\cite{salas:calculus}. According to Zorgkaartnederland a fifth of the complaints in nursing homes come from staff not being able to have enough personal attention for their clients. 
Therefore, lack of staff leads to the less important work not being done. As mentioned communication with clients is a part of the so called ?less important work?. Instead the staff only has time to do administration, helping clients in need and walking around handing the clients their medicine. But there?s no time to have a social conversation.
The only solution for the lack of staff in the world of technology is to reduce the work they have. This way their work stress is reduced and more time for clients will be created. Handing the clients their medicine takes up a lot of time of the staff. They have to go by every client and hand them their medicine, this also causes a great chance of clients getting the wrong medicines. So instead we use medicine dispensers in client?s rooms\cite{bowman:reasoning}. These will be filled once a week by the staff, it will notify when the client needs to take their medicine and notify the staff if the medicines are taken.

\section{Solution}
The solution to help solve this problem is the medicine dispensers. The staff will have to go by every client to fill the dispensers once a week. The dispensers can have a set timer to dispense the medicine or staff can remotely dispense the medicine. The dispensers do not apply to the clients which are not able to take their medicine on their own. For these clients the regular way will be used. So the staff will still have to go to these clients to make sure their medicine is taken. 
The dispenser has a chamber for every day and the chamber has 3 layers on its own. A layer for medicine in the morning, afternoon and evening. The dispenser has an easy to use and small interface to set the time for each day and layer of the chamber. 
When this time comes the chamber and the first layer opens to dispense the medicine into a capture bin. A small speaker and a sign will let the client know the medicine is ready to be taken. If the medicine isn?t taken the sound will go off again and the sign will stay on. If the medicine is still not taken the staff will be notified. Once the medicine is taken the staff will be notified as well.
The medicine itself will be packed in a baxter roll\cite{braams:babel}. This is a product made to transport medicine. 

\section{Innovation}
This product is not entirely new. Candy vending machines are a precursor of the the medicine dispenser. An existing dispenser is enhanced in a way to make it functional in a nursing home. As mentioned in last chapter the dispenser will have sound to notify the clients. Besides notifying the clients it also notifies the staff when the medicine is either taken or not. This means the medicine dispenser is connected over a local network\cite{clark:pct}. We keep this to a local network to offer the best security. The pro?s and cons will be discussed in the next chapter. The dispenser will have a small user interface to set the time to release the medicine. 
The technology company Phillips has made a similar product called the automated medicine dispenser\cite{herlihy:methodology}. This product is for home use and has some cons to it. For a starter your information about your medicine, if the medicine has been taken and also when the medicine is taken is send to Phillips. This data is all being stored by Phillips. The whole use a of the system goes through Phillips. Our medicine dispenser is a stand alone device which will first of all not send any data to any company but it will also not store it, as there is no need to. Also the product made by Phillips has to be activated by the company every time it gets plugged out. It does have some very smart functions which will be implemented in our product.

\section{Evaluation}
For a starter, the medicine dispenser will reduce the chance of handing clients the wrong medicine. Normally the staff will have to come by everyday to give the medicine. With the medicine dispenser this is reduced to once a week\cite{Lamport:LaTeX}. Another simple but time taking job which normally the staff has to do is dealt with by the dispenser. The dispenser does not only hand the medicine but also notifies the staff if the medicine is take. Normally the staff would need to check if the medicine is taken, this is now simply done by the dispenser.
Doing these simple but time taking tasks of the staff creates more time to have a chat with the clients, something which is necessary but not done due to the lack of staff.
A downside to the dispenser is the chance of failure, this could lead to not releasing the medicine. The advantage of a human giving the medicine is that the chance of the staff not going by a clients room and giving the medicine is smaller. To prevent this from happening or prevent this from being a major problem, the medicine dispenser notifies the staff when there is an error or failure. In this way the staff knows there is an error or failure and is able to fix this problem quickly.

\section{Conclusion}
The global idea is to have a medicine dispenser in every room which releases the medicine when needed. This does the staff?s job of going by every room every day. The staff will only need to go by the rooms once a week to fill the dispensers. The dispensers will release the medicine, notify the staff if the medicine is taken and also notify the staff if the medicine is not taken.
 In conclusion, this will solve the problem of not having the time to socialize with the clients. The dispensers will take over some simple, time taking jobs of the staff. With this time the staff can have the much needed social time to chat with the clients.
\newpage

\bibliography{mybib}
\bibliographystyle{apacite}	


\end{document}